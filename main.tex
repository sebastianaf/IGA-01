\documentclass[a4paper,man,natbib]{apa6}

\usepackage[spanish]{babel}
\usepackage[utf8x]{inputenc}
\usepackage{amsmath}
\usepackage{graphicx}
\usepackage[colorinlistoftodos]{todonotes}

\shorttitle{Tarea 1 - TGS y Ensayo}

\begin{document}

\begin{titlepage}
    \begin{center}
        \vspace*{1cm}

        \begin{Huge}
            \shorttitle{Tarea 1 - TGS y Ensayo}
        \end{Huge}

        \includegraphics[width=0.22\textwidth]{img/universidadDelValle.png}
        
        \vfill
        \textbf{Tarea 1 - TGS y Ensayo}\\
        \vfill
        
        Estudiantes\\
        Diego Fernando Chaverra Castillo 201940322\\
        Juan Camilo Santa Gomez 201943214\\
        David Salazar Rodríguez 202181473\\
        Sebastián Afanador Fontal 201629587\\
        \vfill
        Profesora\\
        Diana Marcela Mendoza. Mgs
        
        
        \vfill

        \textbf{Introducción a la Gestión Ambiental}
        
        \vfill
           
        Facultad de Ingeniería\\
        Escuela de Ingeniería de Recursos Naturales y del Ambiente\\
        Universidad del Valle\\
        Cali - Colombia\\
        \vfill
        29 de Abril del 2022

    \end{center}
\end{titlepage}

\section{Primera parte}
\subsection{1. ¿Qué es un sistema?}
Un sistema es una colección de elementos interrelacionados, y por lo tanto cada acción de uno de estos elementos afecta al resto de los elementos actuando como un todo. \citep{von1976teoria}


\subsection{2. Tipos de sistemas}
\subsubsection{Según su interacción con el medio}
\begin{itemize}
    \item Sistema abierto:\\
    Es un sistema que tiene interacción con el ambierte intercambiando energía, recursos o información con el medio externo\citep{10.2307/j.ctv1228hsw}
    \item Sistema cerrado:\\
    Aquel sistema que no interactúa con el ambiente, puede intercambiar información, pero no materia \citep{10.2307/j.ctv1228hsw}    
\end{itemize}

\subsubsection{Según su concepción}
\begin{itemize}
\item Sistemas artificiales:\\
    Sistemas creados por la humanidad, que de no ser por ella, no existirían \citep{10.2307/j.ctv1228hsw}
\item Sistemas naturales:\\
    Son sistemas que existen en la naturaleza que no han sido intervenidos por el hombre.
\end{itemize}

\subsubsection{Según sus elementos}
\begin{itemize}
    \item Sistemas concretos:\\ 
    Aquellos sistemas en donde los elementos que lo conforman se pueden percibir a través de los sentidos.

    \item Sistemas abstractos:\\
    Son sistemas que solo existen a través de un modelo en el mundo de las ideas, generalmente son casos de estudio \citep{10.2307/j.ctv1228hsw}

    \item Sistemas determinísticos:\\ 
    Sistemas en donde los elementos que lo conforman se relacionan de la misma forma a través del tiempo \citep{10.2307/j.ctv1228hsw}

    \item Sistemas probabilísticos:\\ 
    Sistemas en donde sus elementos no poseen una relación clara \citep{10.2307/j.ctv1228hsw}

    
\end{itemize}

\subsection{3. ¿Qué es la Teoría General de Sistemas?}
la Teoría General
de Sistemas es una amalgama de conocimientos que trata de la consideración
global de los fenómenos que estudia, por contraposición al estudio de las partes
para comprender el todo, que es la forma como la ciencia tradicional nos ha
enseñado a desarrollar el conocimiento. \citep{von1976teoria}

\subsection{4. ¿Cuáles son los principios básicos de la T.G.S?}
\subsubsection{Entorno}
Son todas las influencias que son externas al sistema y que condicionan su funcionamiento interno. \citep{10.2307/j.ctv1228hsw}

\subsubsection{Sinergia}
Se refiere a que la suma de las partes de un sistema es menor o diferente que la totalidad porque la interacción entre estas aporta un valor adicional al sistema. Al examinar las partes por individual no podremos describir el comportamiento del sistema o predecirlo.\citep{10.2307/j.ctv1228hsw}

\subsubsection{Resiliencia}
Es la capacidad que tiene un sistema para tolerar los cambios que le produce su entorno.

\subsubsection{Recursividad}
Es la capacidad de un sistema de estár compuesto por partes que a su vez son sistemas, se dice que estos son subsistemas \citep{10.2307/j.ctv1228hsw}

\subsubsection{Retroalimentación}
Es la propiedad que tienen los sistemas de incorporar sus resultados para acciones futuras, este principio permite la amplificación de una entrada \citep{10.2307/j.ctv1228hsw}

\subsubsection{Frontera}
Este principio hace referencia al límite que tiene un sistema para identificar lo que pertenece o no a él. 

\subsubsection{Homeóstasis}
Es la característica que tiene un sistema de autoregularse por medio de mecanismos que pretenden llevarlo a su estado original \citep{10.2307/j.ctv1228hsw}

\subsubsection{Equifinalidad}
Este principio hace énfasis en la importancia de la estructura de los sistemas y no en las condiciones iniciales de su creación, el estado de un sistema depende mas de los elementos que lo conforman que de su estado inicial. \citep{ossa}

\subsubsection{Entropía}
Este principio establece que los sistemas cerrados desaparecerán por la tendencia a el aumento del desorden \citep{10.2307/j.ctv1228hsw}

\subsubsection{Neguentropía}
Es el principio antagónico de la entropía, son los recursos que el sistema importa del entorno para mantener su orden \citep{10.2307/j.ctv1228hsw}


\subsubsection{Emergencia}
Consiste en todos lo elementos y propiedades que componen al sistema cuando este se encuentra en interacción, visto de otro modo, los productos emergentes de un sistema no pueden evidenciarse cuando este no tiene interacción. \citep{ossa}

\section{Segunda parte}
\subsection{1. ¿Qué es un ensayo?}
Es un tipo de texto que permite exponer, evaluar y analizar un tema específico de forma libre y personal; Aunque el ensayo le permite al autor expresarse de manera libre y dar su opinión y punto de vista subjetivo, existen tres tipos de ensayos, los cuales son: Argumentario, expositivo y crítico. \citep{vasquez2005preguntele}
\subsection{2. ¿Cuáles son sus características?}
El ensayo es un género literario libre, aun así, cada uno de los diferentes ensayos tiene unas características que de cierta manera deben respetarse, tales como la libertad temática, el cual el autor puede escoger sobre qué tópico desea escribir, orden argumentativo libre, extensión variada y bibliografía.
\subsection{3. ¿Qué partes debe contener un ensayo?}
\begin{itemize}
    \item Introducción:\\ Aquí se presenta el tema que se expondrá, se dan los antecedentes y el autor presenta su opinión al respecto.
    \item Desarrollo: \\ salen los principales argumentos, y se conoce más a fondo el tema en general, las comparaciones, los datos, etc.
    \item Conclusión: \\ Como último punto, el autor refuerza su opinión, dando así respuesta a alguna de las preguntas iniciales o se crean nuevas incógnitas tras recopilar toda la información.
    \item Bibliografía: \\ Aunque no haga parte del contenido específico, esta parte es fundamental de ser necesaria o de haberse usado alguna información externa.
\end{itemize}

\subsection{4. Extraiga de un ensayo cualquiera un fragmento en el cual se aprecien claramente las características de un ensayo.}

Parece un lugar común de nuestros días mencionar la importancia de una conciencia ecológica, o sea, de una actitud responsable respecto del medio ambiente, a la hora de pensar las dinámicas de producción del mundo actual y el inmediatamente venidero. Sin embargo, nada podría resultar más urgente, dadas las terribles consecuencias climáticas y ambientales que nuestro actual modelo industrial tiene a mediano plazo.\\

La producción y la rentabilidad que alguna vez guiaron el emprendimiento y la innovación contemporánea ahora deben ceder su lugar a la sustentabilidad y la limpieza ecológica, factores que sin embargo encuentran una inmensa oposición de parte de amplios sectores de la sociedad. \\
Esto se debe en parte a la natural resistencia al cambio de los seres humanos y también a la falta de una campaña eficaz de concientización ecológica.\\
\citep{ensayoEjemplo}

\bibliography{references}

\end{document}

%
% Please see the package documentation for more information
% on the APA6 document class:
%
% http://www.ctan.org/pkg/apa6
%